\documentclass[10pt]{article}
\usepackage[utf8]{inputenc}

\usepackage{amsmath}
\usepackage{breqn}
\usepackage{mathtools}
\usepackage{geometry}
\geometry{
    a4paper,
    total={170mm,275mm}
}
\usepackage{multicol}
\usepackage{listings}
\setlength\parindent{0pt}

\lstset{
    language=Python,
    basicstyle=\ttfamily, 
    keywordstyle=\bfseries,
    breaklines=true,
    breakatwhitespace=false
}

\pagenumbering{gobble}

\title{Marker Code Trace Reconstruction}
\author{Wan Yiliang}
\date{September 2022}

\begin{document}


\begin{center}
\hrule
\vspace{.4cm}
{\bf {\Huge Marker Code Trace Reconstruction}}
\vspace{.2cm} \\
Wan Yiliang  (A0250640R) 
\end{center}
\vspace{.1cm}
\hrule

\section{Forward-Backward Algorithm}

  \subsection{Forward Messages}

    \begin{equation}
    \begin{aligned}
      f^{M}(i, j) \,&=\, P(s_1...s_j, \pi^M(i, j) | \mathbf{t}) \\
      f^{I}\,\,(i, j) \,&=\, P(s_1...s_j, \pi^I\,\,(i, j) | \mathbf{t}) \\
      f^{D}\,(i, j) \,&=\, P(s_1...s_j, \pi^D\,(i, j) | \mathbf{t})
    \end{aligned}
    \end{equation}

    \begin{equation} \label{eq1}
    \begin{aligned}
      f^{M}(i, j) \
        &= P(\pi^M(i, j), s_1...s_j | \mathbf{t}) = \sum\nolimits_{\pi_{prev}} P(s_1...s_j, \pi^M(i, j), \pi_{prev} | \mathbf{t}) \\
        &= P(s_1...s_j, \pi^M(i, j), \pi^M(i-1, j-1) | \mathbf{t}) + P(\pi^I(i-1, j-1), \pi^M(i, j), s_1...s_j | \mathbf{t})  \\
        &\qquad + P(s_1...s_j, \pi^M(i, j), \pi^D(i-1, j-1) | \mathbf{t})
    \end{aligned}
    \end{equation}

    First, we examine the first term of (\ref{eq1})

    
    \begin{equation} \label{eq2}
    \begin{aligned}
      P(\pi&^M(i-1, j-1), \pi^M(i, j), s_1...s_j | \mathbf{t}) \\
        & = P(s_1...s_{j-1} | s_j, \pi^M(i-1, j-1), \pi^M(i, j), \mathbf{t}) \times P(s_j | \pi^M(i-1, j-1), \pi^M(i, j), \mathbf{t}) \\
          &\qquad \times P(\pi^M(i, j) | \pi^M(i-1, j-1), \mathbf{t}) \times P(\pi^M(i-1, j-1) | \mathbf{t}) \\
        & = P(s_1...s_{j-1} | \pi^M(i-1, j-1) \mathbf{t}) \times P(s_j | \pi^M(i, j), \mathbf{t}) \times P(\pi^M(i, j) | \pi^M(i-1, j-1), \mathbf{t})\\
          &\qquad \times P(\pi^M(i-1, j-1) | \mathbf{t}) \\
        & = P(s_1...s_{j-1} | \pi^M(i-1, j-1) \mathbf{t}) \times P(\pi^M(i-1, j-1) | \mathbf{t}) \times P(\pi^M(i, j) | \pi^M(i-1, j-1), \mathbf{t}) \\
          &\qquad \times P(s_j | \pi^M(i, j), \mathbf{t}) \\
        & = P(s_1...s_{j-1}, \pi^M(i-1, j-1) | \mathbf{t}) \times P(\pi^M(i, j) | \pi^M(i-1, j-1)) \times\ 
        P(s_j | \pi^M(i, j), \mathbf{t}) \\
        & = f^{M}(i-1, j-1) \times T_{MM} \times Memission[i](s_j)
    \end{aligned}
    \end{equation}

    Similar procedures can be applied to the second and the third term of (\ref{eq1})
    \begin{equation} \label{eq3}
      P(\pi^I(i-1, j-1), \pi^M(i, j), s_1...s_j | \mathbf{t}) = f^{I}(i-1, j-1) \times T_{IM} \times Memission[i](s_j)
    \end{equation}

    \begin{equation} \label{eq4}
      P(\pi^D(i-1, j), \pi^M(i, j), s_1...s_j | \mathbf{t}) = f^{D}(i-1, j-1) \times T_{DM} \times Memission[i](s_j)
    \end{equation}
    
    Put (\ref{eq2}) (\ref{eq3}) and (\ref{eq4}) into (\ref{eq1}), and we can obtain the recursion of $f^{M}$
    \begin{equation}
      f^{M}(i, j) = (f^{M}(i-1, j-1) \times T_{MM}  + f^{I}(i-1, j-1) \times T_{IM} + f^{D}(i-1, j-1) \times T_{DM}) \times Memission[i](s_j)
    \end{equation}

    Similarly, we can obtain the recursion formula of $f^{I}$ and $f^{D}$

    \begin{align}
      &\begin{aligned}
        f^{I}(i, j) &= \sum\nolimits_{\pi_{prev}} P(\pi_{prev}, \pi^I(i, j), s_1...s_j | \mathbf{t}) \\
                    &= (f^{M}(i, j-1) \times T_{MI}  + f^{I}(i, j-1) \times T_{II} + f^{D}(i, j-1) \times T_{DI}) \times Iemission(s_j)
      \end{aligned} \\
      &\begin{aligned}
        f^{D}(i, j) &= \sum\nolimits_{\pi_{prev}} P(\pi_{prev}, \pi^D(i, j), s_1...s_j | \mathbf{t}) \\
                    &= f^{M}(i-1, j) \times T_{MD}  + f^{I}(i-1, j) \times T_{ID} + f^{D}(i-1, j) \times T_{DD}
      \end{aligned}
    \end{align}

  \subsection{Backward Messages}
    \begin{equation}
    \begin{aligned}
      b^{M}(i, j) \,&=\, P(s_{j+1}...s_N | \pi^M(i, j), \mathbf{t}) \\
      b^{I}\,\,(i, j) \,&=\, P(s_{j+1}...s_N | \pi^I\,\,(i, j), \mathbf{t}) \\
      b^{D}\,(i, j) \,&=\, P(s_{j+1}...s_N | \pi^D\,(i, j), \mathbf{t})
    \end{aligned}
    \end{equation}

    \begin{equation} \label{eq121}
    \begin{aligned}
      b^M(i, j) \
        &= P(s_{j+1}...s_N | \pi^{M}(i, j), \mathbf{t}) = \sum\nolimits_{\pi_{next}} P(s_{j+1}...s_N, \pi_{next} | \pi^{M}(i, j), \mathbf{t}) \\
        &= P(s_{j+1}...s_N, \pi^M(i+1, j+1) | \pi^M(i, j), \mathbf{t}) + P(s_{j+1}...s_N, \pi^I(i, j+1) | \pi^M(i, j), \mathbf{t})  \\
        &\qquad + P(s_{j+1}...s_N, \pi^D(i+1, j)) | \pi^M(i, j), \mathbf{t})
    \end{aligned}
    \end{equation}

    Again, $b^{M}(i, j)$ is the sum of three terms which correspond to three possible next states. Examine the first term of (\ref{eq121}).

    \begin{equation} \label{eq122}
    \begin{aligned}
      P(&s_{j+1}...s_N, \pi^M(i+1, j+1) | \pi^M(i, j), \mathbf{t}) \\
        &= P(s_{j+2}...s_N | s_{j+1}, \pi^M(i+1, j+1), \pi^M(i, j), \mathbf{t}) \
          \times P(s_{j+1}, \pi^M(i+1, j+1) | \pi^M(i, j), \mathbf{t}) \\
        &= P(s_{j+2}...s_N | s_{j+1}, \pi^M(i+1, j+1), \pi^M(i, j), \mathbf{t}) \
          \times P(s_{j+1} | \pi^M(i+1, j+1) \pi^M(i, j), \mathbf{t}) \\
          &\qquad\times P(\pi^M(i+1, j+1) | \pi^M(i, j), \mathbf{t}) \\
        &= P(s_{j+2}...s_N | \pi^M(i+1, j+1), \mathbf{t}) \
          \times P(s_{j+1} | \pi^M(i+1, j+1), \mathbf{t}) \
          \times P(\pi^M(i+1, j+1) | \pi^M(i, j)) \\
        &= b^M(i+1, j+1) \times Memission[i+1][s_{j+1}] \times T_{MM}
    \end{aligned}
    \end{equation}

    Similar procedures can be applied to the second and the third term of (\ref{eq121})

      \begin{align}
        P(s_{j+1}...s_N, \pi^I(i, j+1) | \pi^M(i, j), \mathbf{t}) \
          &= b^I(i, j+1) \times Iemission[s_{j+1}] \times T_{MI} \label{eq123}\\
        P(s_{j+1}...s_N, \pi^D(i+1, j)) | \pi^M(i, j), \mathbf{t}) \
          &= b^D(i+1, j) \times T_{MD} \label{eq124}
      \end{align}

    Put (\ref{eq122}) (\ref{eq123}) and (\ref{eq124}) into (\ref{eq121}), and we can obtain the recursion of $b^{M}$

    \begin{equation}
    \begin{aligned}
      b^M(i, j) \
        &= b^M(i+1, j+1) \times Memission[i+1][s_{j+1}] \times T_{MM} \
          + b^I(i, j+1) \times Iemission[s_{j+1}] \times T_{MI} \\
          &\qquad+ b^D(i+1, j) \times T_{MD}
    \end{aligned}
    \end{equation}

    Similarly, we can obtain the recursion formula of $b^{I}$ and $b^{D}$

    \begin{equation}
    \begin{aligned}
      b^I(i, j) \
        &= b^M(i+1, j+1) \times Memission[i+1][s_{j+1}] \times T_{IM} \
          + b^I(i, j+1) \times Iemission[s_{j+1}] \times T_{II} \\
          &\qquad+ b^D(i+1, j) \times T_{ID}
    \end{aligned}
    \end{equation}

    \begin{equation}
    \begin{aligned}
      b^D(i, j) \
        &= b^M(i+1, j+1) \times Memission[i+1][s_{j+1}] \times T_{DM} \
          + b^I(i, j+1) \times Iemission[s_{j+1}] \times T_{DI} \\
          &\qquad+ b^D(i+1, j) \times T_{DD}
    \end{aligned}
    \end{equation}

\end{document}
